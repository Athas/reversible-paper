\documentclass[10pt]{sigplanconf}

%%%%% YOUR PACKAGES %%%%%
\usepackage{amsmath}
\usepackage[utf8x]{inputenc}

%%%%% END %%%%%



\begin{document}
\conferenceinfo{DIKU workshop on Topics in Programming Languages,}{June 2011.} 



%%%%% YOUR TITLE %%%%%
\title{Reversible programming constructs in a non-reversible programming language}
%\subtitle{Subtitle Text, if any}

%%%%% YOUR NAMES %%%%%
\authorinfo{Troels Henriksen \and Philip Carlsen}
           {DIKU, Department of Computer Science, University of Copenhagen}
           {athas@sigkill.dk \and plcplc@gmail.com}
%\authorinfo{Name3}
%           {Affiliation3}
%           {Email3}

\maketitle


%%%%% YOUR ABSTRACT %%%%%
\begin{abstract}
  We investigate the embedding of reversible fragments in a
  non-reversible general-purpose programming language, with the
  primary motivation being the expressivity provided by the guarantees
  of reversibility, not the possibility of execution on reversible
  hardware.  We make use of the type system to enforce separation of
  reversible and nonreversible functions.
\end{abstract}


%%%%% YOUR KEYWORDS %%%%%
% 3 to 5 general words (or topics like "reversible computing") that relates most to the content.
\keywords
reversible computing, haskell, domain-specific language, quantum computing, lambda calculus


%%%%% YOUR CONTENT %%%%%
\section{Introduction}

Reversible circuits hold the promise of fundamentally improving the
energy efficiency of computers, and much effort has been spent on
constructing physical devices capable of performing computation in
reverse.  In parallel, programming languages for the writing of
reversible programs have been developed, although they have not yet
seen much traction for general-purpose programming.  We conjecture
that one part of the reason may be that fully reversible programs can
be somewhat cumbersome to write, and that reversible computers have
not yet materialised to a degree that the effort is rewarded by the
promises of lower power consumption.  On the other hand,
\cite{YokoyamaGlueck:2007:Janus} shows, by the development of a
program for simulating the Schrödinger equation, that reversible
computation can have immediate benefits even on conventional hardware.
We investigate how reversible constructs have been embedded in a
general-purpose nonreversible language, specifically reversible
quantum operations in a library for doing quantum computing in
Haskell, the tradeoffs necessary for the technique to be efficient
(ie. avoiding a full computation history), and compare the result to
Janus.

\section{Section 2}
Reversible computing is interesting and builds on work by
Bennett~\cite{Bennett:1982}. Applications of these principles can be found
in~\cite{ThomsenAxelsen:2009:PPL,YokoyamaGlueck:2007:Janus}.


\section{A reversible $\lambda$-calculus}
\cite{huelsbergen1996logically} defines the lambda calculus equivalent of a
turing machine with a history tape, by means of a $\lambda$ interpreter machine written in SML called SECD-H.


\section{Conclusion}




%%%%% YOUR ACKNOWLEDGEMENTS %%%%%
\acks
Acknowledgments, if needed.



%%%%% YOUR BIBLIOGRAPHY %%%%%
\bibliographystyle{abbrvnat}
% The file where you have bibliography: bibliography.bib
\bibliography{bibliography}

%%%%% YOUR APPENDIX %%%%%
\appendix
\section{Appendix Title}

This is the text of the appendix, if you need one.



\end{document}

%-----------------------------------------------------------------------------
%
%               Template for TPL course using the sigplanconf LaTeX Class
%
% Name:         ACMsigplan-template.tex
%
% Purpose:      A template for sigplanconf.cls, which is a LaTeX 2e class
%               file for ACM SIGPLAN conference proceedings.
%
%-----------------------------------------------------------------------------


\documentclass[10pt]{sigplanconf}

%%%%% YOUR PACKAGES %%%%%
\usepackage{amsmath}

%%%%% END %%%%%



\begin{document}
\conferenceinfo{DIKU workshop on Topics in Programming Languages,}{June 2011.} 



%%%%% YOUR TITLE %%%%%
\title{Reversible programming constructs in a non-reversible programming language}
%\subtitle{Subtitle Text, if any}

%%%%% YOUR NAMES %%%%%
\authorinfo{Troels Henriksen \and Philip Carlsen}
           {DIKU, Department of Computer Science, University of Copenhagen}
           {athas@sigkill.dk \and plcplc@gmail.com}
%\authorinfo{Name3}
%           {Affiliation3}
%           {Email3}

\maketitle


%%%%% YOUR ABSTRACT %%%%%
\begin{abstract}
This is the text of the abstract.
\end{abstract}


%%%%% YOUR KEYWORDS %%%%%
% 3 to 5 general words (or topics like "reversible computing") that relates most to the content.
\keywords
reversible computing, haskell, domain-specific language, quantum computing, lambda calculus


%%%%% YOUR CONTENT %%%%%
\section{Introduction}

The text of the paper begins here.



\section{Section 2}
Reversible computing is interesting and builds on work by Bennett~\cite{Bennett:1982}. Applications of these principles can be found in~\cite{ThomsenAxelsen:2009:PPL,YokoyamaGlueck:2007:Janus}.

\subsection{A subsection}


\subsubsection{A subsubsection}


\section{Conclusion}




%%%%% YOUR ACKNOWLEDGEMENTS %%%%%
\acks
Acknowledgments, if needed.



%%%%% YOUR BIBLIOGRAPHY %%%%%
\bibliographystyle{abbrvnat}
% The file where you have bibliography: bibliography.bib
\bibliography{bibiography}

%%%%% YOUR APPENDIX %%%%%
\appendix
\section{Appendix Title}

This is the text of the appendix, if you need one.



\end{document}

\documentclass[10pt]{sigplanconf}

%%%%% YOUR PACKAGES %%%%%
\usepackage{amsmath}
\usepackage{verbatim}
\usepackage[utf8]{inputenc}

%%%%% END %%%%%



\begin{document}
\conferenceinfo{DIKU workshop on Topics in Programming Languages,}{June 2011.} 



%%%%% YOUR TITLE %%%%%
\title{Reversible programming constructs in a non-reversible programming language}
%\subtitle{Subtitle Text, if any}

%%%%% YOUR NAMES %%%%%
\authorinfo{Troels Henriksen \and Philip Carlsen}
           {DIKU, Department of Computer Science, University of Copenhagen}
           {athas@sigkill.dk \and plcplc@gmail.com}
%\authorinfo{Name3}
%           {Affiliation3}
%           {Email3}

\maketitle


%%%%% YOUR ABSTRACT %%%%%
\begin{abstract}
  We investigate the embedding of reversible fragments in a
  non-reversible general-purpose programming language, with the
  primary motivation being the expressivity provided by the guarantees
  of reversibility, not the possibility of execution on reversible
  hardware.  We make use of the type system to enforce separation of
  reversible and nonreversible functions.
\end{abstract}


%%%%% YOUR KEYWORDS %%%%%
% 3 to 5 general words (or topics like "reversible computing") that relates most to the content.
\keywords
reversible computing, haskell, domain-specific language, quantum computing, lambda calculus


%%%%% YOUR CONTENT %%%%%
\section{Introduction}

Reversible circuits hold the promise of fundamentally improving the
energy efficiency of computers, and much effort has been spent on
constructing physical devices capable of performing computation in
reverse.  In parallel, programming languages for the writing of
reversible programs have been developed, although they have not yet
seen much traction for general-purpose programming.  We conjecture
that one part of the reason may be that fully reversible programs can
be somewhat cumbersome to write, and that reversible computers have
not yet materialised to a degree that the effort is rewarded by the
promises of lower power consumption.  On the other hand,
\cite{YokoyamaGlueck:2007:Janus} shows, by the development of a
program for simulating the Schrödinger equation, that reversible
computation can have immediate benefits even on conventional hardware.
Furthermore, investigations into reversible algorithms may benefit
from being able to leverage those properties of existing languages
(such as data types) that are orthogonal to reversibility.  We show
how reversible constructs can be embedded in a general-purpose
nonreversible language, inspired by reversible quantum operations in a
library for doing quantum computing in Haskell, the tradeoffs
necessary for the technique to be efficient (ie. avoiding a full
computation history), and compare the result to Janus.

\section{Section 2}
Reversible computing is interesting and builds on work by
Bennett~\cite{Bennett:1982}. Applications of these principles can be found
in~\cite{ThomsenAxelsen:2009:PPL,YokoyamaGlueck:2007:Janus}.


\section{A reversible $\lambda$-calculus}
\cite{huelsbergen1996logically} defines the lambda calculus equivalent of a
turing machine with a history tape, by means of a lambda calculus interpreter
machine called SECD-H, written in SML.

SECD-H is structured such that no parts of it ever irreversibly erases any
data, and because of this it is argued that is holds the potential of being a
way to execute execute lambda calculus programs on reversible hardware without
emission of heat due to information erasure.
This is acheived by logging the entire execution history.

If this were to be used as a reversible component in an otherwise
non-reversible programming language, one would either have to store the
execution history together with the result of the reversible computation in
order to be able to later reconstruct the initial state, or just reverse the
history straight away to free the space used by the history log. The latter
option will probably be more desirable in most cases, as the initial arguments
are always available unless specifically thrown away, and the history must
contain the initial arguments as well. Thus, the advantages of this approach
are that of being able to run lambda calculus programs (and by extension, also
programs in most functional languages) without releasing heat from information
erasure on hardware that supports it, at the expense of a guaranteed overhead
in terms of both space and time of at least a factor of two.

\section{Embedded reversible computation in Haskell}

In this section we will explore an existing implementation of
reversible operations and analyse its suitability for our needs.  This
implementation is written in the pure functional programming language
Haskell, although our observations do not depend on any of the unusual
features of Haskell.

As part of the quantum computing library, QIO, presented in
\cite{altenkirchquantum}, a facility for expressing reversible
(\textit{unitary}) is presented.  QIO models reversible computations
as monoids of operations that change the state of the (quantum)
machine, with the identity element being an empty computation, and the
binary operation being sequential computation.  A discussion of
quantum computing is outside the scope of this paper, and we will
restrict ourselves to those parts of QIO that are relevant to the
subject of embedded reversible computing.

The QIO model is close to the one promoted by Janus, and in turn to
reversible hardware itself, although this does not make it much more
realistic to port the library to run on real reversible hardware (as
it is still heavily dependent on the underlying Haskell runtime).  In
this model, the data size (intuitively, number of variables) is
static, and the program is a transformation from one assignment of
values to another.  Although somewhat cumbersome to program, this is
the approach chosen by Janus, and as it seems to be the most
well-understood method in the literature, we choose to preserve it in
our own library.

A number of primitive operations are provided, including negation (for
qubits), introducing local variables, and performing conditional
branching.  In order for the latter to be reversible, neither branch
may change the state used in the conditional, a limitation born from
the choice to use neither computation history or postconditional
assertions in the branch.  This is a serious limitation on the power
of the reversible language, and while it may be acceptable in a
subpart of a greater whole (such as the whole quantum computing
library), we cannot accept such a limitation.

QIO has only a single native data type --- the qubit --- and while it
has a facility for creating other types, such as integers, these are
required to be represented in terms of qubits.  This is appropriate
when the library is intended to model a specific machine at a low
level, but unsuitable for our needs, as we explicitly disavow
considerations of physical implementations of reversibility.  Thus, we
wish for a way to make use of native language data types, for example
integers, floats and strings, as well as an easy way to transfer
values from the non-reversible parts of the program to the parts
implemented via reversible operations.

\section{Conclusion}

We have constructed a language similar to Janus that can make some use
of Haskell functions and data types, although usage is still somewhat
cumbersome, and arrays are not implemented.  The language is useful
for prototyping reversible algorithms and studying reversible
programming techniques, and the implementation is easy to extend and
suitable for experimenting with language extensions.  Work
continues...

%%%%% YOUR ACKNOWLEDGEMENTS %%%%%
%\acks
%Acknowledgments, if needed.



%%%%% YOUR BIBLIOGRAPHY %%%%%
\bibliographystyle{abbrvnat}
% The file where you have bibliography: bibliography.bib
\bibliography{bibliography}

%%%%% YOUR APPENDIX %%%%%
\appendix
\section{Appendix Title}

\subsection{Library source code}

\tiny{
\verbatiminput{src/Reversible.hs}
}


\subsection{Fibonacci program}

\tiny{
\verbatiminput{src/test.hs}
}

\end{document}
